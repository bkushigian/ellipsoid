\section{Termination}
Finally, we would like to know if our algorithm terminates. If we ever find a
point \(a_k \in P\) then we return, so we are concerned with the case where
\(a_k\) is never found to be in \(P\). It turns out we can derive an upper bound
\(N\) such that, if after considering \(E(A_N, a_N)\) we still have not found an
\(a_k \in P\) we may conclude that \(P\) is empty; this follows from a volume
argument. We offer the following two lemmas without proof.\\

\begin{lemma}\label{lem:shrink}
\[\frac{\vol(E_{k+1})}{\vol{(E_k)}} =
\left(\left(\frac{n}{n+1}\right)^{n+1}\left(\frac{n}{n-1}\right)^{n-1}\right)^{1/2}
< e^{-1/2n} < 1\]
\end{lemma}

\begin{lemma}\label{lem:bound}
If \(P\) is a full-dimensional non-empty bounded polytope then
\[\vol(P) \geq 2^{-(n+1)\encode C + n^3}\]
\end{lemma}

The above lemmas offer a shrinkage rate and a lower bound on \(P\)'s volume.
Since Lemma~\ref{lem:shrink} guarantees that \(\vol(E_k) \to 0\) as \(k \to
\infty\), and since Lemma~\ref{lem:bound} offers us a positive lower bound on
the volume, the loop invariant that \(E_k\) contains \(P\) guarantees that after
some finite number of steps we will have found a point in \(P\) or that \(P\)
is empty.
