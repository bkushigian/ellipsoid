
\section{Encoding Lengths}

The \textbf{encoding length} of a number is roughly just the size of the input
in our encoding. While there are many possible encodings, we will assume a
binary encoding. \\

In our binary representation we want to count the number of bits needed to
represent an integer in one's complement, which is captured defining \(\encode
\cdot\) to be the following:
\begin{equation} \label{eq:encode-length-int}
  \encode{n} := 1 + \lceil \log_2(|n| + 1)\rceil,\ n \in \Z
\end{equation}

This may be extended to rational numbers \(r \in \Q\) by considering \(r\)'s
unique coprime representation \(r = p/q,\ \gcd(p, q) = 1\); for rational \(r\)
we define the encoding length to be
%
\begin{equation} \label{eq:encode-length-rational}
\encode{r} := \encode{p} + \encode{q}
\end{equation}
Similarly, the encoding length of vectors and systems of equations can be built
by summing the encoding lengths of the entries.

We list a lemma that will be used later.
\begin{lemma}{Encoding length inequalities}
  \begin{enumerate}[(a)]
    \item For every rational \(r\), \(|r| \leq 2^{\encode r - 1}\).
    \item For every vector \(x \in \Q^n\), \(\norm{x} \leq \norm{x}_1 \leq 2^{\encode x - n} - 1\).
    \item For every vector \(D \in \Q^{n\times n}\), \(|\det D| \leq 2^{\encode D - n^2} - 1\).
  \end{enumerate}
\end{lemma}

\begin{sidenotebox}{\textbf{Sidenote} --- A disagreement on encoding lengths}
There is a historical curiosity that maybe bears mentioning. The
publication of the ellipsoid method brought to the fore disagreements about what
should be counted in the encoding length of a problem. Consider a linear
programming problem given by a matrix \(A \in \Q^{m \times n}\) and vectors \(b
\in \Q^m, c \in \Q^n\). The question is whether the number \(n \cdot m\) should
be considered as the size of the instance, or whether the space needed to encode
\(A\), \(b\), and \(c\) should be counted as well. These lead to different
complexities.
\end{sidenotebox}
