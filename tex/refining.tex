
\section{Refining our Search}
The ellipsoid method is best thought of as a variant of binary search.  In
binary search we are guaranteed that the inputs are sorted according to some
ordering, and we are able to efficiently split our problems into subproblems of
equal size. This splitting into subproblems is significantly more difficult in
the ellipsoid method: for binary search we simply computed the midpoint of an
array \(m = \lfloor (h - l) / 2 \rfloor\), and considered the indices greater
than \(m\) and the indices less than \(m\) as two subproblems.

\todo{Fix above, just a sketch/beginning for the motivation for ellipsoid}

\subsection{Cuts}
In our binary search analogy, the center of the ellipsoid being contained in
polytope \(P\) is the action that corresponds to choosing the midpoint of the
array. If the center \(a\) of the ellipse \(E(A,a)\) is contained in \(P\),
which may be readily checked, then we are done. Otherwise we want to produce a
subproblem.\\

By convexity, there must be a line through the center \(a\) of \(E\) that does
not intersect \(P\), and we may cut \(E\) in half, producing convex body \(E'\).
Here we need to resolve two things:
\begin{enumerate}
  \item What line do we choose (and how do we find it?)
  \item Which side of the line does our polytope \(P\) end up upon?
\end{enumerate}

The answer to the first helps us answer the second.\\

Suppose that \(a \not\in P\). This means that there was some violation of a
constraint \(c^Tx \leq \gamma\), so that \(c^Ta > \gamma\). Thus the hyperplane
\[\{x : c^T x = c^T a\}\]
through the center \(a\) of \(E\) cuts \(E\) in half, and by construction \(P\)
must be contained in the half
\begin{equation} \label{eq:ellipsoid-cut}
E \cap \{x : c^T x \leq c^T a\}.
\end{equation}
\todo{Image}

\subsection{L\"owner John Ellipsoids}
Unfortunately, dividing ellipsoid \(E\) in half in (\ref{eq:ellipsoid-cut})
doesn't yield a new ellipsoid, so we don't yet have a new subproblem. Instead,
we would like to find a new ellipsoid that is smaller than the first that
contains the half ellipsoid \(E'\). 

\begin{theorem} For every \(K \subseteq \R^n\) there exists a unique ellipsoid
\(E\) of minimal volume containing \(K\).
\end{theorem}


\begin{defbox}
\begin{definition}[L\"owner John Ellipsoid]
The \textbf{L\"owner John} ellipsoid of a convex body \(K\) is the smallest
ellipsoid that contains 
\end{definition}


\end{defbox}

After we've made our cut of \(E\), resulting in \(E'\), we want to find the
L\"owner John ellipsoid of \(E'\). In general the L\"owner John ellipsoid of an
arbitrary convex body \(k\) is very difficult to compute, but there are special
cases that can be computed easily. The formula is somewhat involved and isn't
particularly illumniating so we we don't offer it here \todo{maybe below, in a
later part}

