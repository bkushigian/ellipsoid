\section{Initialization of the Ellipsoid Method}
No that we have a basic idea of how the ellipsoid method works, we would like to
initialize the first ellipsoid \(E_0\). We want to maintain the loop invariant
that \(P \in E_k \implies P \in E_{k+1}\), so \(E_0\) should contain \(P\). But
\(P\) is specified by some arbitrary integral system of inequalities \(Cx \leq
d\), so how can we do this? Couldn't this correspond to an arbitrary polytope?\\

It turns out that we do have some bounds on \(P\) in relation to the encoding
length of \(C\) and \(d\). \\

\begin{defbox}
  \begin{definition}
    The \textbf{encoding length} of an integer \(m\) is the number of bits
    needed to represent \(m\), which can be calculated to be \(1 + \lceil
    \log_2(|m|)\rceil\). The \textbf{encoding length} of integral vector \(v\)
    is the sum of the encoding lengths of the components of \(v\), namely
    \(\encode v = \sum_{i=1}^n \encode v_i\). Similarly, the \textbf{encoding
    length} of an integral matrix \(A\) is the sum of the encoding length of the
    \(k\) column vectors of \(C\), namely \(\encode C = \sum_{i=1}^k \encode
    c_i\).
  \end{definition}
\end{defbox}

We denote by \(\encode{C, d}\) the encoding length of \(C\) and \(d\), which is
simply \(\encode C + \encode d\). Next we provide a lemma that gives us an
initial ellipsoid.

\begin{lemma}\label{lemma:encoding-length-radius}
  \(P \subseteq \ball(R, 0)\), where 
  \begin{equation}\label{eq:encoding-length-radius}
    R := \sqrt{n}2^{\encode {C, d} - n^2}.
  \end{equation}
  \begin{proof}
      We are assuming that \(P\) is bounded and full-dimensional in \(\R^n\);
      that is, that it is the convex hull of its \(n+1\) vertices.  Suppose we
      have some bound \(t\) such that for each component \(v_i\) of each vertex
      \(v\) of \(P\) we had \(|v_i| \leq t\).  Then the furthest a \(v \in P\)
      could be from the origin would be \(v = (t,t,t,t,\ldots, t)\), which would
      have distance \(\sqrt{t^2 + t^2 + \cdots + t^2} = \sqrt{n} t\).\\

      From polyhedral theory we know that a vertex \(v\) is the
      \textit{unique} solution to some system of \(n\) inequalities, say
      \[Bx \leq b\]
      where \(B\) is an \(n\times n\) submatrix of input matrix \(C\), and \(b\)
      us a subvector of input bounds-vector \(d\). Since \(v\) is the unique
      solution to \(Bx \leq b\), \(B\) is a non-singular matrix and we can apply
      Cramer's rule to conclude that
      %
      \[v_i = \frac{\det{B_i}}{\det{B}},\]
      %
      where \(B_i\) is the matrix obtained by replacing the \(i\)th column of
      \(B\) with \(b\), whence 
      %
      \[|v_i| = \frac{|\det{B_i}|}{|\det{B}|}.\]
      %
      Since \(B\) is non-singular and integral we have \(|\det{B}| \geq 1\), and
      \(|v_i| \leq |\det B_i|\), giving us a bound on \(|v_i|\) as desired.\\

      We now appeal to a lemma, offered without proof, that states that \(|\det
      B_i| \leq 2^{\encode {B_i} - n^2} - 1\). But since \(B_i\) has only
      entries from \(C\) and \(d\) we must have \(\encode{B_i} \leq
      \encode{C,d}\), and we conclude that
      \[|v_i| \leq 2^{\encode{B_i} - n^2} - 1 \leq 2^{\encode{C,d} - n^2}.\]
      Noting that this upper bound is universal for all components of all
      vertices of \(P\), we conclude that \(P \subset \ball(R, 0)\).
  \end{proof}
\end{lemma}

