\section{Initialization of the Ellipsoid Method}
Our second motivating question is ``how do we find an initial ellipsoid''? For
the basic ellipsoid method we are assuming that \(P\) is bounded and
full-dimensional, but it turns out that neither of these assumptions are needed
in the fully general case.\\

In a perfect world we would have explicit bounds for each of our \(x_i\), say
\(l_i \leq x_i \leq u_i\). If we do have this then we can take \(R\) to be
%
\[R := \sqrt{\sum_{i=1}^n \max\{u_i^2, l_i^2\}}.\]
%
Unfortunately the world isn't perfect\footnote{as evidenced by Emacs users (go
Vim)} and we need a more general approach. In this case we use the
\textit{encoding length} of \(C\) and \(d\), denoted \(\encode{C, d}\).
\todo{offer proof? this seems cool and I'd like to understand it better.}

The \textbf{encoding length} of a number is roughly just the size of the input.
While there are many possible encodings, we will assume a binary encoding.
Further, for simplicity of exposition, we are assuming that all inputs are
integral so that we don't need to worry about the encoding size of floating
point numbers.\\

\begin{sidenotebox}{\textbf{Sidenote} --- A disagreement on encoding lengths}
However, there is a historical curiosity that maybe bears mentioning. The
publication of the ellipsoid method brought to the fore disagreements about what
should be counted in the encoding length of a problem. Consider a linear
programming problem given by a matrix \(A \in \Q^{m \times n}\) and vectors \(b
\in \Q^m, c \in \Q^n\). The question is whether the number \(n \cdot m\) should
be considered as the size of the instance, or whether the space needed to encode
\(A\), \(b\), and \(c\) should be counted as well. These lead to different
complexities.
\end{sidenotebox}

This gives us the following definition of \(\encode \cdot \):
\begin{equation} \label{eq:encode-length-int}
  \encode{n} := 1 + \lceil \log_2(|n| + 1)\rceil,\ n \in \Z
\end{equation}

This may be extended to rational numbers \(r \in \Q\) by considering \(r\)'s
unique coprime representation \(r = p/q,\ \gcd(p, q) = 1\); for rational \(r\)
we define the encoding length to be
%
\begin{equation} \label{eq:encode-length-rational}
\encode{r} := \encode{p} + \encode{q}
\end{equation}
Similarly, the encoding length of vectors and systems of equations can be built
by summing the encoding lengths of the entries.

We list a lemma that will be used later.
\begin{lemma}{Encoding length inequalities}
  \begin{enumerate}[(a)]
    \item For every rational \(r\), \(|r| \leq 2^{\encode r - 1}\).
    \item For every vector \(x \in \Q^n\), \(\norm{x} \leq \norm{x}_1 \leq 2^{\encode x - n} - 1\).
    \item For every vector \(D \in \Q^{n\times n}\), \(|\det D| \leq 2^{\encode D - n^2} - 1\).
  \end{enumerate}
\end{lemma}
