\section{Introduction}
Let \(P = \{x : Bx \leq b\}\) be a polytope.
\begin{question}
  Is \(P\) empty? If not, give a point in \(P\).
\end{question}

The ellipsoid method gives us a way to find this in polynomial time. Let's take
a look at an example.  \textbf{\color{red}Do rough (non-technical) example here.}

\begin{example}
  \def\A{\left(\begin{matrix} 1 & 0 \\ -1 & 0 \\ 0 & 1 \\ 0 & -1 \end{matrix}\right) }
  \def\b{\left(\begin{matrix} 1\\ 0\\ 1\\ 0 \end{matrix}\right)}
  Suppose we have the bounds \(x \geq 0, x \leq 1, y \geq 0, y \leq 1\), which
  we represent by

  \[\A x \leq \b \]
  
  This defines the unit square, which you and I know, but our computer hasn't
  figured this out yet, so we would like to help it along. Our approach is as
  follows:
  \begin{enumerate}
    \item \textbf{Find some ellipsoid \(E\) (which we leave undefined for now---think of
      a shmooshed shircle or sphere) that must contain any point \(y \in P\).}
      Note that we haven't assumed that there \textit{is} a point in \(P\), only
      that \textit{if} there's a point in \(P\) then that point must necessarily
      lie in \(E\).

    \item \textbf{Until we are confident that we can quit}
      \begin{enumerate}
        \item Check if the center of the ellipse satisfies all constraints in
          \(P\).
        \item If the center of \(P\) is contained in polytope \(P\) then
          \(P\) is nonempty, and our computer can report this.

        \item Otherwise, the center is not contained in \(P\), and we must look
          closer. We can draw a hyperplane through the center of \(E\) parallel
          to the violated constraint and chop the ellipse in half.

        \item We can find a new, smaller, ellipse around this new convex body
          and we start the loop over again.
      \end{enumerate}
  \end{enumerate}
\end{example}


There are some new concepts that we just introduced without motivation. First is
that of an ellipsoid. 
