\section{Introduction}
Let \(P = \{x : Cx \leq b\}\) be a polytope.
\begin{question}
  Is \(P\) empty? If not, give a point in \(P\).
\end{question}

The ellipsoid method gives us a way to find this in polynomial time. Let's take
a look at an example.  \textbf{\color{red}Do rough (non-technical) example here.}

\begin{examplebox}{A motivating example of the Ellipsoid Method}
\begin{example}
  \def\A{\left(\begin{matrix} -1 & -1 \\ 1 & 1 \\ 1 & -1 \\ -1 & 1 \end{matrix}\right) }
  \def\b{\left(\begin{matrix} -3\\ 5\\ 1\\ 1 \end{matrix}\right)}
  Suppose we have the bounds

  \[\A x \leq \b.\]
  
  The set of solutions \(P\) is the diamond with length-two diagonals centered at
  \((2,2)\). We would like a procedure to determine if \(P\) is empty. We work
  as follows:
  \begin{enumerate}
    \item \textbf{Find some ellipsoid \(E\) that contains any point \(x \in P\).}
      We haven't defined ellipsoids yet, so for now just picture a shmooshed
      circle. Also, note that we haven't assumed that there \textit{is} a point
      in \(P\), only that \textit{if} there's a point in \(P\), that point
      must necessarily lie in \(E\). \todo{Picture}

    \item \textbf{Until we are confident that we can quit}
      \begin{enumerate}
        \item Check if the center of the ellipsoid satisfies all constraints in
          \(P\).
        \item If the center of \(E\) is contained in polytope \(P\) then
          \(P\) is nonempty and we return.

        \item Otherwise, the center is not contained in \(P\), and we must
          refine our search by picking a smaller ellipsoid.
      \end{enumerate}
  \end{enumerate}
\end{example}
\end{examplebox}


This brings up several new motivating questions:
\begin{enumerate}
  \item What is an ellipsoid?
    
  \item How can we find an ellipsoid to contain \(P\)?

  \item How can we refine our search?

  \item How do we know when we can stop?
\end{enumerate}


