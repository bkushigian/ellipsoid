%----------------------------------------------------------------------------------------
%	PACKAGES AND OTHER DOCUMENT CONFIGURATIONS
%----------------------------------------------------------------------------------------

\documentclass[fontsize=12pt]{scrartcl} % A4 paper and 11pt font size

\usepackage[utf8]{inputenc}
\usepackage[T1]{fontenc} % Use 8-bit encoding that has 256 glyphs
\usepackage[english]{babel} % English language/hyphenation
\usepackage{amsmath,amsfonts,amsthm} % Math packages
\usepackage{sectsty} % Allows customizing section commands
\usepackage{mathtools}
\usepackage{centernot}
\usepackage{color}
\usepackage{tcolorbox}
\usepackage{bbm}
\usepackage{amssymb}
\usepackage{enumerate}
\usepackage{fancyhdr} % Custom headers and footers
\usepackage{geometry}
\usepackage{mathpazo}

\def\R{\mathbb R}
\def\RS{\mathbb S}
\def\F{\mathbb F}
\def\Z{\mathbb Z}
\def\Q{\mathbb Q}
\def\C{\mathbb C}
\def\D{\mathbb D}
\def\H{\mathbb H}

%%% MACROS THAT WE SHOULD USE
%%% For balls (this represents S(a,b) in the text
\def\ball{\mathcal{B}}
%%% For volumnes
\def\vol{\textnormal{vol}}
%%% For the encoding length: <C, d> is written \encode{C, D}
\newcommand{\encode}[1]{\langle {#1} \rangle}
%%% for various todos, this appears in red text in the pdf
\newcommand{\todo}[1]{{\color{red}\textbf{(todo:} \textit{{#1}}\textbf{)}}}
%%%
\newcommand{\norm}[1]{\left\lVert#1\right\rVert}

%%% For making definitions
\newtcolorbox{defbox}{colback=red!6!white,colframe=red!75!black}
%%% For making examples
\newtcolorbox{examplebox}[1]{colback=blue!4!white,colframe=blue!50!black,title=#1}
%%% For making sidenotes
\newtcolorbox{sidenotebox}[1]{colback=green!4!white,colframe=green!50!black,title=#1}


%THEOREMS
\theoremstyle{definition}
\newtheorem{question}{Q}
\newtheorem{example}{Example}
\newtheorem*{lemma}{Lemma}
\newtheorem*{fact}{Fact}
\newtheorem{definition}{Definition}
\newtheorem{theorem}{Theorem}

\newcommand{\includecode}[2][Python]{\lstinputlisting[caption=#2, escapechar=, style=custom#1]{#2}}

\newcommand{\owl}{\widehat{\dbinom{\odot_\text{v}\odot}{\wr}}}

\allsectionsfont{\centering \normalfont\scshape} % Make all sections centered, the default font and small caps

\pagestyle{fancyplain} % Makes all pages in the document conform to the custom headers and footers
\fancyhead{} % No page header - if you want one, create it in the same way as the footers below
\fancyfoot[L]{} % Empty left footer
\fancyfoot[C]{} % Empty center footer
\fancyfoot[R]{\thepage} % Page numbering for right footer
\renewcommand{\headrulewidth}{0pt} % Remove header underlines
\renewcommand{\footrulewidth}{0pt} % Remove footer underlines
\setlength{\headheight}{13.6pt} % Customize the height of the header

\numberwithin{equation}{section} % Number equations within sections (i.e. 1.1, 1.2, 2.1, 2.2 instead of 1, 2, 3, 4)
\numberwithin{figure}{section} % Number figures within sections (i.e. 1.1, 1.2, 2.1, 2.2 instead of 1, 2, 3, 4)
\numberwithin{table}{section} % Number tables within sections (i.e. 1.1, 1.2, 2.1, 2.2 instead of 1, 2, 3, 4)

\setlength\parindent{0pt} % Removes all indentation from paragraphs - comment this line for an assignment with lots of text
\newcommand{\horrule}[1]{\rule{\linewidth}{#1}} % Create horizontal rule command with 1 argument of height

